\section{Purpose of This Document}

This document aims to serve as an aid to executing initial missions
on the MIT REMUS-100 platform at the MIT Sailing Pavilion. It 
accompanies a simple "Hello World!" set of mission files.

\section{The Alvin Mission}

The Alvin mission is a simple mission for execution at the MIT
Sailing Pavilion. It consists of a few simple waypoints with 
some safety measures. The idea is shown in Figure \ref{fig_alvin_one}.
The large polygon represents a safety op-box. In the figure the 
simulated UUV has reached its south-west waypoints and is heading
toward its final north-east waypoint.

\vspace{0.15in}
\begin{figure}[H]
\centering 
\includegraphics[width=.82\linewidth]{figures/alvin_one.eps}
\caption{{\bf The Alvin Mission}: The Alvin mission consists of three
  waypoints at a depth of 1.2 meters.}
\label{fig_alvin_one}
\end{figure}
\vspace{0.1in}

\section{Accessing the Required Repositories}

There are three repositories (two sets of software, one set of mission
files) needed to run the mission described in this document. (1) The 
"moos-ivp" public tree, (2) The "moos-ivp-local" public+MIT tree, 
and (3) the set of mission files in the "mission-alpha" tree. All 
three are available on the oceanai SVN server. With the exception of
the Acomms and REMUS payload interface code, all are publicly available. 

\pskip

It is assumed that a user account has been made on the REMUS payload
computer named \var{ruser}, short for ``robot user''. This is a
generic account that has certain read-only privileges on the oceanai
server. There is also a generic \var{iver} account for similar usage
on the Iver2 UUVs. We may make a generic \var{remus} account if
desired.

\subsection{Accessing the MOOS-IvP Tree}

The software required to run the Alvin mission {\em in simulation} is
completely contained in the publicly available MOOS-IvP tree.

\footnotesize
\begin{verbatim}
  www.moos-ivp.org/download.html
\end{verbatim}
\normalsize

This software should be downloaded directly on to the REMUS payload
computer under the \var{ruser} generic account. I recommend downloading
the SVN trunk, but the latest stable release of MOOS-IvP is another 
option. This tree includes the MOOS code from Oxford, and should be
built prior to building the moos-ivp-local tree described next. To 
build the moos-ivp tree fully, including the GUI related applications, 
one would need to install the graphics related libraries described
in the README.txt file. These are unnecessary on the UUV, and the
build can be configured to not build these applications. Alternatively
one can launch the build script with the \var{-k} option which is 
the standard make option to skip over un-built targets to build all
later targets. 


\subsection{Accessing the MOOS-IvP Local Tree}

To run the Alvin mission on the REMUS, additional modules from the
moos-ivp-local tree need to be installed on the vehicle. This includes:

\begin{itemize}
\item {\em iREMUS}: The MOOS payload autonomy interface to the
  front-seat computer.
\item {\em pAcommsHandler}: The MOOS interface to the WHOI micromodem
\end{itemize}

Both the iREMUS and pACommsHandler modules are not in the public
portion of the moos-ivp-local tree, but are available to any member of
the MIT community. There are a fair amount of dependencies for the
pAcommsHandler module. The README files should be consulted, or the
author, Toby Schneider, tes@mit.edu. The Acomms software is needed 
to run a simple mission kicked off with a command over the acoustic
modem. Altenatively, the first simple mission may be kicked off with 
a command over Wifi, using pMOOSBridge. In the latter case, the issue
of building pAcommsHandler and producing a working Acomms message-set 
configuration file could be avoided for now.

\subsection{Accessing the Alvin Mission Files}

The Alvin mission is available in a separate tree on the oceanai SVN
server.

\footnotesize
\begin{verbatim}
  svn co svn+ssh://18.38.2.158/home/svnp/mission-alvin
\end{verbatim}
\normalsize

\noindent
The current admin configuration of this tree allows read-write
access to anyone with the user account on the oceanai machine with 
the generic ``svnuser'' group privileges. This includes the \var{ruser}
account. Therefore an operator editing mission files on the REMUS
may be able to check in changes to the SVN server directly.

%%--------------------------------------------------------------------
\section{An Overview of the Alvin Mission Directory}

A fresh checkout of the mission-alvin tree should produce a directory
with the following files:

\footnotesize
\begin{verbatim}
 $ cd mission-alvin/
 $ ls
    alvin.bhv
    alvin.moos
    alvin_remus.moos
    clean.sh
    launch.sh
    pavilion.info
    pavilion.tif
\end{verbatim}
\normalsize

\noindent 
The items of primary interest are the \var{.bhv} and \var{.moos}
files, discussed further below. The \var{alvin.bhv} file should serve
as the helm configuration file for both simulation and field
operation.  the \var{alvin.moos} file is the MOOS mission
configuration file for simulation and the \var{alvin\_remus.moos} file
is the MOOS configuration file for field operation. The latter two
files have some overlapping components, but differ substantially as
the simulation MOOS modules are swapped out for other modules in the
\var{alvin\_remus.moos} file.

\pskip

The \var{launch.sh} script is a handy script for launching the alvin
simulation. It is a convention used in other example missions distributed
and documented on the moos-ivp.org site. It is particularly handy for 
launching simulations with multiple UUVs, but in this simple mission
it can be used to launch the simulation with a MOOS time warp:

\footnotesize
\begin{verbatim}
 $ cd mission-alvin/
 $ ./launch 10
\end{verbatim}
\normalsize

\noindent
The \var{clean.sh} script is the complementary script for cleaning up
temporary or log files generated during simulation. The pavilion.tif
and pavilion.info files provide the image data used by the pMarineViewer
GUI application shown in Figure \ref{fig_alvin_one}. More info can be
found on this in \cite{ben10} or \cite{bnsl10a}.

%%-----------------------------------------------------------------
\subsection{The Alvin Behavior Configuration File}

The helm, pHelmIvP module for the Alvin example mission is configured
in \var{alvin.bhv}. It consists of a very simple mode space
(\var{SURVEYING} or \var{INACTIVE}), and three simple behaviors. The
file is short and given entirely in Listing \ref{lst_alvin_bhv} 
below. 

\vspace{0.1in}
\refstepcounter{listing}
\vspace{0.1in}
\noindent {\em {Listing \arabic{listing} - The alvin.bhv configuration
file for the IvP Helm.}}
\label{lst_alvin_bhv}
\scriptsize
\begin{verbatim}
   0  //------------------------------------------------
   1  // File: alvin.bhv
   2  // Name: M. Benjamin
   3  // Date: Nov 26th, 2010
   4  //------------------------------------------------
   5  
   6  initialize  DEPLOY = false
   7  
   8  set MODE = SURVEYING {
   9    DEPLOY = true
  10  } INACTIVE
  11  
  12  //------------------------------------------------
  13  Behavior = BHV_Waypoint
  14  { 
  15    name      = waypt_survey
  16    pwt       = 100
  17    condition = MODE == SURVEYING
  18    endflag   = DEPLOY=false
  19    endflag   = MOOS_MANUAL_OVERIDE=true
  20
  21            lead = 8
  22           speed = 1.0   // meters per second
  23          radius = 4.0
  24       nm_radius = 10.0
  25          points = -20,-28:-16,-37:20,-22
  26  }
  27  
  28  //------------------------------------------------
  29  Behavior = BHV_ConstantDepth
  30  {
  31    name       = const_depth
  32    pwt        = 100
  33    condition  = MODE == SURVEYING
  34    duration   = no-time-limit3
  35  
  36         depth = 1.2
  37  }
  38  
  39  //------------------------------------------------
  40  Behavior = BHV_OpRegion
  41  {
  42    name       = op_region
  43    pwt        = 100
  44    condition  = MODE == SURVEYING
  45  
  46               max_time = 600
  47                polygon = -111,-49:-45,-182:181,-79:144,3:84,35
  48     trigger_entry_time = 2.0
  49      trigger_exit_time = 1.0
  50  }
\end{verbatim}
\normalsize

The helm's hierarchical mode declarations are declared in lines 8-10.
More can be found on this in \cite{bnsl10a}. The modes are trivial
here and the same could be achieved by simply having the conditions on
lines 17, 33, and 44 be simply \var{condition = DEPLOY=true}. Nevertheless
it is a good convention to use the modes, since the mode configuration
is picked up by the helm to generate helm status reports picked up by 
pMarineViewer and other apps.

\pskip

The Waypoint behavior is configured in lines 12-26 with the three
simple waypoints on line 25. This behavior is not configured to repeat
its points.  When it is finished, it is finished permanently. This can
be changed with the \var{perpetual} flag as described in
\cite{bnsl10a}.  The ConstantDepth behavior is provided to keep the
vehicle at a survey depth of 1.2 meters. The OpRegion behavior in
lines 39-50 provides two safety checks. It limits the total mission
time to 600 seconds and limits the vehicle operation to be within the
polygon given on line 46 and rendered in Figure \ref{fig_alvin_one}.

\pskip

A good safety check would be to alter the \var{max\_time} and the 
\var{polygon} to allow the vehicle to violate their limits and check 
that the helm properly posts the flag \var{IVPHELM\_ENGAGED="DISENGAGED"}, 
and that the \var{iREMUS} module properly reads this handles this 
properly by either bringing the vehicle back to the surface, or handing
control back to the REMUS front-seat computer, or both.


%%-----------------------------------------------------------------
\subsection{The Alvin MOOS Configuration File}

The MOOS community for simulating the Alvin mission is configured in
the \var{alvin.moos} file. It is described below. The head components
are shown in the Listing \ref{lst_alvin_moos_one}

\vspace{0.1in}
\refstepcounter{listing}
\vspace{0.1in}
\noindent {\em {Listing \arabic{listing} - The alvin.moos
    configuration file for the simulated MOOS community.}}
\label{lst_alvin_moos_one}
\scriptsize
\begin{verbatim}
   0  //------------------------------------------------
   1  // File: alvin.moos
   2  // Name: M. Benjamin
   3  // Date: Nov 26th, 2010
   4  //------------------------------------------------
   5  
   6  ServerHost = localhost
   7  ServerPort = 9000
   8  Simulator  = true
   9  
  10  Community    = alvin
  11  MOOSTimeWarp = 1
  12  LatOrigin    = 43.825300 
  13  LongOrigin   = -70.330400 
  14  
  15  //------------------------------------------
  16  // Antler configuration  block
  17  ProcessConfig = ANTLER
  18  {
  19    MSBetweenLaunches = 200
  20  
  21    Run = MOOSDB          @ NewConsole = true
  22    Run = pNodeReporter   @ NewConsole = true
  23    Run = pHelmIvP        @ NewConsole = true
  24    Run = iMarineSim      @ NewConsole = true
  25    Run = pMarinePID      @ NewConsole = true
  26    Run = pMarineViewer   @ NewConsole = true
  27  }
\end{verbatim}
\normalsize

The local datum is set on lines 12-13, and the six MOOS apps 
comprising this simulation are launched via Antler on lines
21-26. See \cite{url_oxmoos} for more on pAntler. The apps in
lines 24-26 would not be running on a fielded vehicle but are
necessary here for simulation.

\pskip

The helm behavior content is configured both in the \var{alvin.bhv}
file, but is also generally configured as a MOOS process in the
\var{alvin.moos} file, as shown in Listing \ref{lst_alvin_moos_two}.
The helm decision domain is configured in lines 10-12, and a pointer
to the behavior configuration file is provided on line 8.

\vspace{0.1in}
\refstepcounter{listing}
\vspace{0.1in}
\noindent {\em {Listing \arabic{listing} - The alvin.moos
    configuration file for the IvP Helm.}}
\label{lst_alvin_moos_two}
\scriptsize
\begin{verbatim}
   0  //------------------------------------------
   1  // pHelmIvP config block
   2  
   3  ProcessConfig = pHelmIvP
   4  {
   5    AppTick    = 4
   6    CommsTick  = 4
   7  
   8    Behaviors  = alvin.bhv
   9    Verbose    = true
  10    Domain     = course:0:359:360
  11    Domain     = speed:0:4:21
  12    Domain     = depth:0:10:101
  13  }
\end{verbatim}
\normalsize

The configuration block for the PID controller, pMarinePID, is
shown in Listing \ref{lst_alvin_moos_three} below. Note the 
configuration for depth control on lines 23-27. This represents
a recent change in the implementation of pMarinePID, and is tied
to the latest version of moos-ivp SVN head r.3014.

\vspace{0.1in}
\refstepcounter{listing}
\vspace{0.1in}
\noindent {\em {Listing \arabic{listing} - The alvin.moos
    configuration file for the pMarinePID application.}}
\label{lst_alvin_moos_three}
\scriptsize
\begin{verbatim}
   0  //------------------------------------------
   1  // pMarinePID config block
   2  
   3  ProcessConfig = pMarinePID
   4  {
   5    AppTick    = 20
   6    CommsTick  = 20
   7  
   8    VERBOSE       = true
   9    DEPTH_CONTROL = true
  10  
  11    // Yaw PID controller
  12    YAW_PID_KP		 = 0.5
  13    YAW_PID_KD		 = 0.0
  14    YAW_PID_KI		 = 0.0
  15    YAW_PID_INTEGRAL_LIMIT = 0.07
  16  
  17    // Speed PID controller
  18    SPEED_PID_KP		 = 1.0
  19    SPEED_PID_KD		 = 0.0
  20    SPEED_PID_KI		 = 0.0
  21    SPEED_PID_INTEGRAL_LIMIT = 0.07
  22  
  23    // Depth PID controller
  24    DEPTH_PID_KP           = 0.5
  25    DEPTH_PID_KD           = 0.0
  26    DEPTH_PID_KI           = 0.0
  27    DEPTH_PID_INTEGRAL_LIMIT = 0.07
  28  
  29    //MAXIMUMS
  30    MAXRUDDER    = 100
  31    MAXTHRUST    = 100
  32    MAXELEVATOR  = 100
  33  
  34    // Non-zero SPEED_FACTOR overrides SPEED_PID
  35    // DESIRED_THRUST = DESIRED_SPEED * SPEED_FACTOR
  36    SPEED_FACTOR = 20
  37  }
\end{verbatim}
\normalsize


\section{Launching the Alvin Example Mission}

The Alvin example mission is configured to launch all MOOS processes
and the helm, with the helm in a stand-by (\var{"INACTIVE"}) mode. 
It can be launched by poking \var{DEPLOY=true} to the MOOSDB. This 
is most easily done by simply clicking the \var{DEPLOY:true} button
on the pMarineViewer display as shown in Figure \ref{fig_alvin_one}.
It can also be done via any other method of poking the MOOSDB, e.g., 
via uPokeDB for example. 

\pskip

The end state of the mission is achieved when the vehicle has
traversed all three waypoints. At this point the waypoint behavior
posts its endflags (line 18-19, Listing \ref{lst_alvin_bhv}), and the
tag next to the vehicle on the GUI should read: \var{"Alvin
  (DISENGAGED)"}, since the helm has become disengaged by posting
\var{MOOS\_MANUAL\_OVERIDE=true}. At this point the helm has also
posted \var{IVPHELM\_ENGAGED=DISENGAGED}, which is the recommended
indicator for the payload interface module (iREMUS) to hand control
back to the front-seat computer.
